
\section{基本规则}

姊姊好!
行文的基本规则和 Markdown 其实差不太多:
一段话内容由若干行连续的文字组成。
两段话之间由至少一个空行分隔开来。
注意你在源文件(\path{xxx.tex})
里加入多少个空行都会被当成一个空行来对待,
这一点对于空格来说也是一样的。
使用 \LaTeX\ 时建议先不要管排版,
所以不要想着要在最后的文档中哪里插入空行或者空格。
你只要专注于你想要写的「内容」上,
而不用担心「格式」上的问题。

看。这就是第二个自然段了。

一些常用的 ``Markup'' 包括%
    \emph{强调}的内容,
    行间\code`代码`,
    以及
    \propernameZH{书名}或\propername{A Book's Name}。

你可以像编程时一样在文档中使用「注释」:
在 \LaTeX\ 中,
我们用一个百分号(\code`%`)作为行注释符。
比如
% 这是一个注释
这里的源代码实际上是:
\begin{lstlisting}[language={[LaTeX]TeX}]
比如
% 这是一个注释
这里的源代码实际上是:
\end{lstlisting}

