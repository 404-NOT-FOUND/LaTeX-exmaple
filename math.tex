
\section{数学公式}

数学公式呢,
其实和上次是一样的
(因为上次就是在用 \LaTeX\ 来编译的数学公式啦)。
再复习一下:
行间的数学内容用一对美元符号(\code`$`)来包围,
比如 $e^{i\pi} + 1 = 0$。
大块的或重要的数学内容单独展示,
此时可以使用基本的 \code`\[` 和 \code`\]` 来包围:
\[
    \log_{10} 10 = 1
\]
或更复杂的 \code `multiline` 来展示多行公式:
\begin{multline*}
    x = 4 \times \Bigl( y \cdot z + \frac{1}{2} c \Bigr) \\
      = 4 y \cdot z + \frac{4}{2} c \\
      = 4 y \cdot z + 2 c \\
      = 4 \times 2^{12} \times 2^{-10} + 2 \times \lg 100 \\
      = 4 \times 2^2 + 2 \times 2 \\
      = 4 \times 16 + 4 \\
      = 64 + 4 \\
      = 68
\end{multline*}
如果你有强迫症、需要将等号对齐,
可以这样写:
\[
    \begin{split}
        x &= 4 \times \Bigl( y \cdot z + \frac{1}{2} c \Bigr) \\
          &= 4 y \cdot z + \frac{4}{2} c \\
          &= 4 y \cdot z + 2 c \\
          &= 4 \times 2^{12} \times 2^{-10} + 2 \times \lg 100 \\
          &= 4 \times 2^2 + 2 \times 2 \\
          &= 4 \times 16 + 4 \\
          &= 64 + 4 \\
          &= 68
    \end{split}
\]

关于括号的大小:
推荐手动选择合适大小的括号。
括号的尺寸小(正常)到大分别为
\begin{itemize}
\item
    不加括号、
\item
    \code`\bigl` 和 \code`\bigr`、
\item
    \code`\Bigl` 和 \code`\Bigr`、
\item
    \code`\biggl` 和 \code`\biggr`、
    以及
\item
    \code`\Biggl` 和 \code`\Biggr`。
\end{itemize}
这些括号的效果如下:
\[
    \Biggl(\biggl(\Bigl(\bigl((x)\bigr)\Bigr)\biggr)\Biggr)
\]

如果你有什么数学符号不知道该怎么打,
可以参考\href{http://web.mit.edu/rsi/www/pdfs/math-symbols.pdf}{这份资料}。

