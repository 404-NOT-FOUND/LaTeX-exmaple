
\section{代码陈列}

代码可以直接写在文档里面,
也可以从一个文件中截取。
下面是直接写在文档中的代码:
\begin{lstlisting}[language=python]
print('hello world')
\end{lstlisting}
这是 \path{code/hello.py} 中的内容:
\lstinputlisting[language=python]{code/hello.py}
下面我们截取其中的第二到三行:
\lstinputlisting[language=python, firstline=2, lastline=3]{code/hello.py}

注意直接写在文档中的代码前面不要留有空格,
否则空格会被保留。
比如
「这样写」
和
                                    「这样写」
在生成的 PDF 中是一样的;
但
\begin{lstlisting}[language=python]
print('hello world')
\end{lstlisting}
和
    \begin{lstlisting}[language=python]
    print('hello world')
    \end{lstlisting}
则是不一样的(代码部分空格会被保留)。

