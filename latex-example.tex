\documentclass{article}

%% Title
%% ----------------------------------------------------------------------------

% 标题信息的设置。并不会真正生成标题
\title{\LaTeX\ 使用简例}
\author{Jy}
\date{\today}

%% Links
%% ----------------------------------------------------------------------------

\usepackage{url}

%% Fonts
%% ----------------------------------------------------------------------------

%% English fonts
\usepackage[no-math]{fontspec}
% \setmainfont[Mapping=tex-text]{Times New Roman}
% \setsansfont[Mapping=tex-text]{Calibri}
% \setmonofont{Courier New}
%% Chinese, needs fontspec
\usepackage[CJKchecksingle]{xeCJK}
\usepackage{CJKnumb}
	\setCJKmainfont[
        BoldFont   = {Microsoft YaHei},
        ItalicFont = {KaiTi}]
        {FangSong}
	\setCJKsansfont{}
	\setCJKmonofont{Microsoft YaHei}
	\punctstyle{hangmobanjiao}

%% Document
%% ----------------------------------------------------------------------------

\begin{document}

\maketitle % 加上这句话才会生成标题


\section{基本规则}

姊姊好!
行文的基本规则和 Markdown 其实差不太多:
一段话内容由若干行连续的文字组成。
两段话之间由至少一个空行分隔开来。
注意你在源文件(\path{xxx.tex})
里加入多少个空行都会被当成一个空行来对待,
这一点对于空格来说也是一样的。
使用 \LaTeX\ 时建议先不要管排版,
所以不要想着要在最后的文档中哪里插入空行或者空格。
你只要专注于你想要写的「内容」上,
而不用担心「格式」上的问题。

看。这就是第二个自然段了。


\section{列表}

下面是几种列表(list)的使用。
一个没有标序号的列表叫「无序列表(unordered list)」:
\begin{itemize}
\item 
    苹果
\item 
    梨
\item 
    香蕉
\end{itemize}
一个标有序号的列表则叫「有序列表(ordered list)」:
\begin{enumerate}
\item 
    苹果
\item 
    梨
\item 
    香蕉
\end{enumerate}
还有一种列表叫描述列表「(description list)」:
\begin{description}
\item[苹果]
    红色
\item[梨]
    浅黄色
\item[香蕉]
    黄色
\end{description}

\end{document}

