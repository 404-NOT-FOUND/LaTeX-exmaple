\documentclass{article}

%% Title
%% ----------------------------------------------------------------------------

% 标题信息的设置。并不会真正生成标题
\title{\LaTeX\ 使用简例}
\author{Jy}
\date{\today}

%% Links
%% ----------------------------------------------------------------------------

\usepackage{url}
\usepackage{cleveref}

%% Tables
%% ----------------------------------------------------------------------------

\usepackage{booktabs}
\newcommand*\botrule{\bottomrule}

%% Fonts
%% ----------------------------------------------------------------------------

%% English fonts
\usepackage[no-math]{fontspec}
% \setmainfont[Mapping=tex-text]{Times New Roman}
% \setsansfont[Mapping=tex-text]{Calibri}
% \setmonofont{Courier New}
%% Chinese, needs fontspec
\usepackage[CJKchecksingle]{xeCJK}
\usepackage{CJKnumb}
	\setCJKmainfont[
        BoldFont   = {Microsoft YaHei},
        ItalicFont = {KaiTi}]
        {FangSong}
	\setCJKsansfont{}
	\setCJKmonofont{Microsoft YaHei}
	\punctstyle{hangmobanjiao}

%% Coding
%% ----------------------------------------------------------------------------

\usepackage{listings}
\def\code{\lstinline}

%% Document
%% ----------------------------------------------------------------------------

\begin{document}

\maketitle % 加上这句话才会生成标题


\section{基本规则}

姊姊好!
行文的基本规则和 Markdown 其实差不太多:
一段话内容由若干行连续的文字组成。
两段话之间由至少一个空行分隔开来。
注意你在源文件(\path{xxx.tex})
里加入多少个空行都会被当成一个空行来对待,
这一点对于空格来说也是一样的。
使用 \LaTeX\ 时建议先不要管排版,
所以不要想着要在最后的文档中哪里插入空行或者空格。
你只要专注于你想要写的「内容」上,
而不用担心「格式」上的问题。

看。这就是第二个自然段了。


\section{列表}

下面是几种列表(list)的使用。
一个没有标序号的列表叫「无序列表(unordered list)」:
\begin{itemize}
\item 
    苹果
\item 
    梨
\item 
    香蕉
\end{itemize}
一个标有序号的列表则叫「有序列表(ordered list)」:
\begin{enumerate}
\item 
    苹果
\item 
    梨
\item 
    香蕉
\end{enumerate}
还有一种列表叫描述列表「(description list)」:
\begin{description}
\item[苹果]
    红色
\item[梨]
    浅黄色
\item[香蕉]
    黄色
\end{description}

\section{图表}

我们也可以插入图片(figure),比如 \cref{fig:attention}。
\begin{figure}[htpb]
    \centering
    \includegraphics[width=0.8\linewidth]{pics/thanks-for-your-attention.png}
    \caption{中国某市某繁华街区十字路口附近的墙上张贴的启事}
    \label{fig:attention}
\end{figure}
写的时候尽量使用 \code`label` 与 \code`cref` 交叉引用的方式来书写。
避免依靠图、表的位置来描述。
即多使用 \code`如 \cref{xxx} 所示` 
而避免使用 \code`如下图所示` 
这样的语言。

要插入一张表(table),就使用 \code`table` 环境。
你可以用 Word 做好表导出成图片再引用进来(如 \cref{tab:wordtab})。
也可以在 \LaTeX\ 中编写表格(如 \cref{tab:latextab}),
其实也很简单。

\begin{table}[htpb]
    \centering
    \caption{%
        这张表是用 Google Doc 制作、再作为图片的形式导入进来的。
    }
    \label{tab:wordtab}
    \includegraphics[width=10cm]{pics/fruit-specs.png}
\end{table}

\begin{table}[htpb]
    \centering
    \caption{%
        这张表是直接使用 \LaTeX\ 编写的。
    }
    \label{tab:latextab}
    \begin{tabular}{clr} % 三列,对齐方式分别为:中C 左L 右R
        \toprule
        水果 & 颜色   & 价格(美元/磅) \\
        \midrule
        香蕉 & 黄色   & 0.69            \\
        梨   & 浅黄色 & 2.83            \\
        苹果 & 红色   & 2134.87         \\
        \botrule
    \end{tabular}
\end{table}

\end{document}

