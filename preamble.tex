
% 这个文件定义了一些版式风格、以及一些命令。
% 你并不需要阅读这些内容。

\documentclass{article}

%% Lists
%% ----------------------------------------------------------------------------

\usepackage{paralist}
\renewenvironment{itemize}{%
    \begin{compactitem}
}{%
    \end{compactitem}
}
\renewenvironment{enumerate}{%
    \begin{compactenum}
}{%
    \end{compactenum}
}
\renewenvironment{description}{%
    \begin{compactdesc}
}{%
    \end{compactdesc}
}

%% Colors
%% ----------------------------------------------------------------------------

\usepackage{xcolor}
\definecolor{lightblue}{RGB}{39, 104, 192}

%% Links
%% ----------------------------------------------------------------------------

\usepackage[
    bookmarksnumbered,
    pdfencoding=auto,
    pdfborder=1,
    breaklinks,
    colorlinks, linkcolor=lightblue, urlcolor=blue
]{hyperref}
\hypersetup{
    pdfmenubar=true,
    citecolor=green,
    filecolor=magenta,
}
\usepackage{url}

\usepackage[
    nameinlink,
    noabbrev,
]{cleveref}

%% Tables
%% ----------------------------------------------------------------------------

\usepackage{booktabs}
\newcommand*\botrule{\bottomrule}

%% Fonts
%% ----------------------------------------------------------------------------

%% English fonts
\usepackage[no-math]{fontspec}
% \setmainfont[Mapping=tex-text]{Times New Roman}
% \setsansfont[Mapping=tex-text]{Calibri}
% \setmonofont{Courier New}
%% Chinese, needs fontspec
\usepackage[CJKchecksingle]{xeCJK}
\usepackage{CJKnumb}
	\setCJKmainfont[
        BoldFont   = {Microsoft YaHei},
        ItalicFont = {SimSun}]
        {FangSong}
	\setCJKmonofont{KaiTi}
	\punctstyle{hangmobanjiao}

%% Coding
%% ----------------------------------------------------------------------------

\usepackage{listings}
\lstset{%
    basicstyle=\ttfamily,
    frame=L,
}

%% Spacing
%% ----------------------------------------------------------------------------

% space of caption and the context;
    \setlength{\belowcaptionskip}{3pt}

\usepackage{setspace}
\onehalfspacing
\setlength{\parindent}{2em}
\addtolength{\parskip}{3pt}
\usepackage{indentfirst}

%% Math
%% ----------------------------------------------------------------------------

\usepackage{amsmath, amssymb}

%% Markup
%% ----------------------------------------------------------------------------

\def\code{\lstinline}
\newcommand*\propername[1]{\textit{#1}}
\newcommand*\propernameZH[1]{《#1》}

